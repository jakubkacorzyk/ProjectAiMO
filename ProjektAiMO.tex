\documentclass[11pt]{article}
\setlength\headheight{13.6pt}%
\usepackage[utf8]{inputenc} % Required for inputting international characters
\usepackage[T1]{fontenc} % Output font encoding for international characters
\usepackage{polski}
\usepackage{mathpazo} % Palatino font
\usepackage{graphicx}
\usepackage{fancyhdr}
\usepackage{etoolbox}
\usepackage{blindtext}
\usepackage{geometry}
\geometry{legalpaper, margin=1in}
\usepackage[cleardoublepage=plain]{scrextend}
\graphicspath{ {images/.\ProjectAiMO} } %change to yours path
\pagestyle{fancy}
\fancyhf{}
\rhead{2017/2018}
\lhead{Projekt z Analizy i Modelowania Oprogramowania}
\rfoot{Strona \thepage}
\begin{document}
	
	\begin{titlepage} 
	
		\newcommand{\HRule}{\rule{\linewidth}{0.5mm}} % Defines a new command 
		
		\center % Centre everything on the page
		
		%------------------------------------------------
		%	Nagłówki
		%------------------------------------------------
		
		\textsc{\LARGE Akademia Górniczo-Hutnicza im. Stanisława Staszica w Krakowie}\\[1.5cm] % Main heading such as the name of your university/college
		
		\textsc{\Large Projekt z Analizy i Modelowania Oprogramowania}\\[0.5cm] % Major heading such as course name
		
		\textsc{\large Informatyka EAIiIB 2017/2018}\\[0.5cm] % Minor heading such as course title
		
		%------------------------------------------------
		%	Tytuł
		%------------------------------------------------
		
		\HRule\\[0.4cm]
		
		{\huge\bfseries Automat do sprzedaży zakąsek}\\[0.4cm] % Title of your document
		
		\HRule\\[1.5cm]
		
		%------------------------------------------------
		%	Authorzy
		%------------------------------------------------
		
		\begin{minipage}{0.4\textwidth}
			\begin{flushleft}
				\large
				\textit{Autorzy}\\
				Jakub \textsc{Kacorzyk} \\
				Bartłomiej \textsc{Łazarczyk}
			\end{flushleft}
		\end{minipage}
		~
		\begin{minipage}{0.4\textwidth}
			\begin{flushright}
				\large
				\textit{Prowadzący}\\
				dr inż. Wojciech \textsc{Szmuc} % Supervisor's name
			\end{flushright}
		\end{minipage}
		
		%------------------------------------------------
		%	Logo
		%------------------------------------------------
		
		\vfill
		\includegraphics[scale=1.0]{logo.jpg}
		
		%------------------------------------------------
		%	Data
		%------------------------------------------------
		
		\vfill\vfill\vfill % Position the date 3/4 down the remaining page
		
		{\large\today} 
		
		%----------------------------------------------------------------------------------------
		
		\vfill % Push the date up 1/4 of the remaining page
		
	\end{titlepage}
	
	\tableofcontents
	\cleardoublepage
	\setcounter{page}{2}
	
	\section{Streszczenie projektu}
		Celem naszego projektu jest zamodelowanie automatu do sprzedaży zakąsek.
		Automat w trybie oczekiwanie wyświetla na wyświetlaczu dotykowym prośbę o wybranie numeru produktu. Klient po wpisaniu numeru produkt otrzymuje informację o dostępności produktu w maszynie oraz jego cenie. Następnie wybiera jedą z dwóch dostępnych metod płatności i płaci za wybrany produkt. Modelowany przez nas automat przyjmuję zapłatę w gotówce (w postaci monet) oraz płatność w terminalu, który jest osobnym systemem. W przypadku płatności bilonem, jeżeli cena była niższa niż kwota znajdującego się w automacie, po otrzymaniu produktu wydawana jest reszta. W przypadku płatności kartą klient podąża za instrukcjami terminalu, który po zakończeniu transakcji przesyła informacje do automatu. Możliwe jest też zrezygnowanie z zakupu i wtedy automat zwraca całą wpłaconą kwotę. Automat podlega także serwisowaniu. Osoba za to odpowidzialna może zmienić cenę produktów, zmienić produkt oraz wybrać gotówkę z automatu.
	\newpage
    \section{Diagram użyć.}
		\begin{center}
			\includegraphics[scale=0.65]{UseCaseDiagram.pdf}
		\end{center}
		\newpage
	\section{Diagramy sekwencji.}
		\subsection{Wybieranie produktu udane.}
		\begin{center}
			\includegraphics[scale=0.65]{WybranieProduktuUdane.pdf}
		\end{center}
		\subsection{Wybranie produktu nieudane.}
		\begin{center}
			\includegraphics[scale=0.65]{WybranieProduktuNieudane.pdf}
		\end{center}
		\newpage
		\subsection{Płatność kartą udana.}
		\begin{center}
			\includegraphics[scale=0.65]{PlatnoscKartaUdana.pdf}
		\end{center}
		\subsection{Płatność kartą nieudana.}
		\begin{center}
			\includegraphics[scale=0.65]{PlatnoscKartaNieudana.pdf}
		\end{center}
		\newpage
		\subsection{Płatność gotówką.}
		\begin{center}
			\includegraphics[scale=0.60]{PlatnoscGotowkaUdana.pdf}
		\end{center}
		\newpage
		\subsection{Wydanie produktu.}
		\begin{center}
			\includegraphics[scale=0.65]{WydanieProduktu.pdf}
		\end{center}
		\newpage
		\subsection{Wydanie reszty.}
		\begin{center}
			\includegraphics[scale=0.65]{WydanieReszty.pdf}
		\end{center}
		\newpage
		\subsection{Obsługa automatu.}
		\begin{center}
			\includegraphics[scale=0.65]{ObslugaAutomatu.pdf}
		\end{center}
		\newpage
		\subsection{Pobranie gotówki.}
		\begin{center}
			\includegraphics[scale=0.65]{PobranieGotowki.pdf}
		\end{center}
		\newpage
		\subsection{Uzupełnienie automatu.}
		\begin{center}
			\includegraphics[scale=0.65]{UzupelnianieAutomatu.pdf}
		\end{center}
		\newpage
		\subsection{Zmiana produktu.}
		\begin{center}
			\includegraphics[scale=0.65]{ZmianaProduktu.pdf}
		\end{center}
	\end{document}